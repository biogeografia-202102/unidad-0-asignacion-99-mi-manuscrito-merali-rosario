\documentclass[11pt,]{article}
\usepackage[left=1in,top=1in,right=1in,bottom=1in]{geometry}
\newcommand*{\authorfont}{\fontfamily{phv}\selectfont}
\usepackage[]{mathpazo}


  \usepackage[T1]{fontenc}
  \usepackage[utf8]{inputenc}



\usepackage{abstract}
\renewcommand{\abstractname}{}    % clear the title
\renewcommand{\absnamepos}{empty} % originally center

\renewenvironment{abstract}
 {{%
    \setlength{\leftmargin}{0mm}
    \setlength{\rightmargin}{\leftmargin}%
  }%
  \relax}
 {\endlist}

\makeatletter
\def\@maketitle{%
  \newpage
%  \null
%  \vskip 2em%
%  \begin{center}%
  \let \footnote \thanks
    {\fontsize{18}{20}\selectfont\raggedright  \setlength{\parindent}{0pt} \@title \par}%
}
%\fi
\makeatother




\setcounter{secnumdepth}{3}

\usepackage{longtable,booktabs}

\usepackage{graphicx,grffile}
\makeatletter
\def\maxwidth{\ifdim\Gin@nat@width>\linewidth\linewidth\else\Gin@nat@width\fi}
\def\maxheight{\ifdim\Gin@nat@height>\textheight\textheight\else\Gin@nat@height\fi}
\makeatother
% Scale images if necessary, so that they will not overflow the page
% margins by default, and it is still possible to overwrite the defaults
% using explicit options in \includegraphics[width, height, ...]{}
\setkeys{Gin}{width=\maxwidth,height=\maxheight,keepaspectratio}

\title{Asociacion y composicion floristica de la familia Sapotaceae en la
parcela permanente de 50h, Isla Barro Colorado  }



\author{\Large Merali Rosario\vspace{0.05in} \newline\normalsize\emph{Afiliación, normalmente algo tal que ``Estudiante, Universidad Autónoma
de Santo Domingo (UASD)''}  }


\date{}

\usepackage{titlesec}

\titleformat*{\section}{\normalsize\bfseries}
\titleformat*{\subsection}{\normalsize\itshape}
\titleformat*{\subsubsection}{\normalsize\itshape}
\titleformat*{\paragraph}{\normalsize\itshape}
\titleformat*{\subparagraph}{\normalsize\itshape}

\titlespacing{\section}
{0pt}{36pt}{0pt}
\titlespacing{\subsection}
{0pt}{36pt}{0pt}
\titlespacing{\subsubsection}
{0pt}{36pt}{0pt}





\newtheorem{hypothesis}{Hypothesis}
\usepackage{setspace}

\makeatletter
\@ifpackageloaded{hyperref}{}{%
\ifxetex
  \PassOptionsToPackage{hyphens}{url}\usepackage[setpagesize=false, % page size defined by xetex
              unicode=false, % unicode breaks when used with xetex
              xetex]{hyperref}
\else
  \PassOptionsToPackage{hyphens}{url}\usepackage[unicode=true]{hyperref}
\fi
}

\@ifpackageloaded{color}{
    \PassOptionsToPackage{usenames,dvipsnames}{color}
}{%
    \usepackage[usenames,dvipsnames]{color}
}
\makeatother
\hypersetup{breaklinks=true,
            bookmarks=true,
            pdfauthor={Merali Rosario (Afiliación, normalmente algo tal que ``Estudiante, Universidad Autónoma
de Santo Domingo (UASD)'')},
             pdfkeywords = {palabra clave 1, palabra clave 2},  
            pdftitle={Asociacion y composicion floristica de la familia Sapotaceae en la
parcela permanente de 50h, Isla Barro Colorado},
            colorlinks=true,
            citecolor=blue,
            urlcolor=blue,
            linkcolor=magenta,
            pdfborder={0 0 0}}
\urlstyle{same}  % don't use monospace font for urls

% set default figure placement to htbp
\makeatletter
\def\fps@figure{htbp}
\makeatother

\usepackage{pdflscape} \newcommand{\blandscape}{\begin{landscape}}
\newcommand{\elandscape}{\end{landscape}} \usepackage{float}
\floatplacement{figure}{H}
\newcommand{\beginsupplement}{ \setcounter{table}{0} \renewcommand{\thetable}{S\arabic{table}} \setcounter{figure}{0} \renewcommand{\thefigure}{S\arabic{figure}} }


% add tightlist ----------
\providecommand{\tightlist}{%
\setlength{\itemsep}{0pt}\setlength{\parskip}{0pt}}

\begin{document}
	
% \pagenumbering{arabic}% resets `page` counter to 1 
%
% \maketitle

{% \usefont{T1}{pnc}{m}{n}
\setlength{\parindent}{0pt}
\thispagestyle{plain}
{\fontsize{18}{20}\selectfont\raggedright 
\maketitle  % title \par  

}

{
   \vskip 13.5pt\relax \normalsize\fontsize{11}{12} 
\textbf{\authorfont Merali Rosario} \hskip 15pt \emph{\small Afiliación, normalmente algo tal que ``Estudiante, Universidad Autónoma
de Santo Domingo (UASD)''}   

}

}








\begin{abstract}

    \hbox{\vrule height .2pt width 39.14pc}

    \vskip 8.5pt % \small 

\noindent Resumen del manuscrito


\vskip 8.5pt \noindent \emph{Keywords}: palabra clave 1, palabra clave 2 \par

    \hbox{\vrule height .2pt width 39.14pc}



\end{abstract}


\vskip 6.5pt


\noindent  \section{Introducción}\label{introducciuxf3n}

La diversidad y estructura de los bosques miden los recursos y la
abundancia en un área geográfica, por ejemplo, los bosques de la familia
Sapotaceae son importantes para proporcionar alimento a las especies de
vida silvestre (Martínez-Sovero et al., 2021). La familia Sapotaceae
está ampliamente distribuida en la zonas tropicales (Smedmark, n.d.).
Produce madera de alta calidad, frutas tropicales y algunas especies
producen látex, siendo una familia de plantas de importancia ecológica y
económica (Martínez-Sovero, Iglesias-Osores, \& Villena-Velásquez,
2020).

La Isla Barro Colorado es una reserva natural ubicada en el lago Gatún
del Canal de Panamá. Debido a su capacidad de investigación, es una de
las regiones tropicales más conocidas en materia de biología y ecología
tropical (``Isla barro colorado y biología tropical,'' n.d.).La isla
exhibe características importantes, tres de las cuales son la
estabilidad ambiental, su ubicación geográfica (en un área de
importancia internacional) y la capacidad para investigar grupos
específicos de organismos (Rodríguez-Flores \& Barrios, 2020). Sin
embargo, no se han hecho estudios completos de la familia Sapotaceae,
donde se estudie los diferentes analisis de ecologia numerica.

El objetivo de este trabajo es determinar la asociacion, composicion
floristica y distribucion de la familia Sapotaceae en la parcela
permanente de 50h de la isla Barro Colorado. Ademas, analizar la
organizacion de las especies en los cuadros de 1 hectárea e identificar
si existe algun patron con alguna variable ambiental, asi como tambien,
explicar si hay especies indicadoras o con preferencia por determinadas
condiciones ambientales. Por otra parte, evaluar si la familia
sapotaceae esta suficientemente representada segun los análisis de
estimación de riqueza, determinar cuales son las variables ambientales
que presentan asociacion con la diversidad alpha y mostrar cuales son
las especies que contribuyen a la diversidad beta. Por ultimo, pero no
menos importante, examinar en un espacio bidimensional las tendencias de
ordenacion de las especies y determinar si las especies presentan patron
aglomerado.

\section{Metodología}\label{metodologuxeda}

\subsection{Area de Estudio}\label{area-de-estudio}

La isla de Barro Colorado es una colina de 1,500 hectáreas ubicada a 137
msnm en el lago Gatún. La parte superior de la isla es ancha y plana, y
se asienta sobre un lecho de roca de basalto, de la cual irradian
colinas empinadas y valles tallados en rocas sedimentarias que contienen
gran cantidad de restos volcánicos. El suelo es arcilloso y la
profundidad varía de 50 cm a un metro. El clima es típico de las areas
tropicales (Pérez et al., 2005; Windsor et al., n.d.). Su vegetación
está formada por bosques semideciduos de tierras bajas, y se han
registrado más de 1,300 especies de plantas vasculares (\emph{Flora de
la isla barro colorado}, n.d.).

La parcela permanente de árboles de 50 hectáreas se estableció en 1980
en el bosque húmedo tropical. El sitio es un rectángulo de 1,000 m de
largo por 500 m de ancho, ubicado en la meseta central de la isla. Está
dividido en 1,250 cuadrantes de 20x20 m, en el cual se han contabilizado
todos los arboles con más de 10 mm de diámetro a la altura del pecho
cada cinco años desde 1985 (R. A. y H. Condit Richard y Chisholm, n.d.;
R. y L. Condit Richard y P ~'e rez, n.d.; Pérez et al., 2005) (ver
figura \ref{fig:mapabarro}).

\begin{figure}
\centering
\includegraphics[width=1.00000\textwidth]{mapa Barro Colorado.png}
\caption{Mapa topográfico de la parcela de cincuenta hectáreas
\label{fig:mapabarro}}
\end{figure}

\subsection{Maeriales y Metodos}\label{maeriales-y-metodos}

Los datos de cada uno de los cuadrantes de una hectárea que componen
BCI, fueron procesados en R (R Core Team, 2020), teniendo en cuenta la
matriz ambiental y la matriz de comunidad, lo cuales contienen datos de
las variables ambientales, tales como condiciones edaficas, tipo de
habitat, topografia del lugar, clasificacion etaria del bosque, y datos
demograficos y geoferenciacion espacial de todos los individuos
censados. Se adaptaron scripts reproducibles recuperados de Batlle
(2020), utilizando la colección de paquetes multifuncionales vegan
(Oksanen et al., 2019), Tidyverse (Wickham, 2017), BiodiversityR (R.
Kindt \& Coe, 2005) y indicspecies (De Caceres \& Legendre, 2009).

Para conocer las características de los datos almacenados de la matriz
de comunidad y ambiental, se realizó un análisis exploratorio que
incluyó visualización de gráficos, tablas, mapas de los cuadrantes de
una hectárea y tablas de correlación lineal entre las dos variables de
la matriz, lo que permitió una vista común y ayudó a determinar
procedimientos más detallados a continuación.

\subsection{Medición de asociación
(ma)}\label{mediciuxf3n-de-asociaciuxf3n-ma}

Para realizar las pruebas de medicion de asociacion, se calculó la
distancia euclidiana entre los cuadrados considerados objetos. Para
ello, se requierió una transformación de la matriz de comunidad mediante
el método Hellinger, que incluye elevar al cuadrado la abundancia
relativa yij (cociente resultante de cada valor de abundancia entre la
suma de los sitios), como muestra la formula \ref{fig:formula}. Donde j
denota cada tipo o columna de la matriz, i es la fila o cuadrante e i+
representa la suma de filas de la matriz de la i-ésima fila (Legendre \&
Gallagher, 2001). Además, se evaluó la distancia euclidiana entre los
cuadrantes en términos de ocurrencia de especies. Se utilizó el índice
de disimilitud de Jaccard de la matriz normalizada para convertir el
valor de abundancia en un valor binario (Brocard, Gillet, \& Legendre,
2018). del mismo modo, se empleó la métrica de Jaccard para aplicar la
transposición de la matriz de la comunidad y convertir a datos Presencia
/ ausencia para medir el grado de asociación entre especies.

\begin{figure}
\centering
\includegraphics[width=1.00000\textwidth]{Formula2.png}
\caption{Formula. Transformación de la matriz de comunidad mediante el
método Hellinger \label{fig:formula}}
\end{figure}

Para poder comparar la relación entre especies en función de su
abundancia, se utilizó estandarización ji-cuadrado de la matriz de
comunidad transpuesta (Legendre \& Gallagher, 2001). Se examinó la
ocurrencia entre especies y su distribución en BCI por el coeficiente de
orrelación entre rangos de Spearman para medir el grado de correlación
entre las variables riqueza númerica de especies y la abundancia con las
variables ambientales geomorfológicas, y la composición química del
suelo(Brocard et al., 2018).

\subsection{Analisis de agrupamiento}\label{analisis-de-agrupamiento}

El método jerarquico aglomerativo de asociación entre pares de
cuadrantes (según la composición de especies) bajo el estándar de enlace
completo, y el método de Ward basado en la varianza mínima, se utilizan
como método preliminar para el análisis de agrupamiento, con el fin de
probar su efectividad en lograr un grupo consistente de importancia
ecológica (Brocard et al., 2018). Luego, estos generaron dendrogramas
que posteriormente son comparados con la matriz de distancia de cuerdas
(Legendre \& Gallagher, 2001). Usando correlación cofenética entre los
dos para determinar el número ideal de grupos. Además, se utilizó
remuestreo bootstrap y boostrap multiescalar para conocer la
probabilidad de éxito de la inferencia del número de grupos y la
identidad de sus componentes (Brocard et al., 2018). Las distribuciones
se basaban en una probabilidad de 91\% o más de acierto para el método
bootstrap y de un 95\% para boostrap multiescalar.

Dado que se localizaron patrones consistentes en la composicion y numero
de grupos entre los metodos examinados, los análisis de agruoamiento
posteriores se basan en los que se produce por enlace completo e incluye
dos grupos compuestos por 20 y 30 cuadrantes, respectivamente(ver figura
\ref{fig:Dendograma}).

\begin{figure}
\centering
\includegraphics[width=1.00000\textwidth]{Dendograma.png}
\caption{Dendrogramas de los grupos producidos por Ward y Complete
\label{fig:Dendograma}}
\end{figure}

Para conocer las especies distintas o asociadas a cada grupo, se utilizo
el valor del indicador o índice IndVal (``Conjuntos de especies y
especies indicadoras,'' n.d.), basado en permutaciones aleatorias de los
sitios según la presencia de especies y la abundancia de estos. De
manera similar, el grado de asociación de una especie con una
preferencia particular por el grupo de cuadrantes considerado grupo en
estudio, expresado como el coeficiente de correlación biserial puntual
(Brocard et al., 2018). Se adoptó un enfoque similar al anterior a lo
largo de las pruebas estadísticas de la hipótesis nula, basada en las
especies presentes en los cuadrantes pertenecientes a un determinado
grupo realizado al azar. Esta prueba se hizo reordenando aleatoriamente
los valores de abundancia y comparando sus distribuciones con las
obtenidas previamente (Cáceres \& Legendre, 2009).

\subsection{Análisis de diversidad
alpha}\label{anuxe1lisis-de-diversidad-alpha}

La diversidad alpha representa la diversidad de especies a lo largo de
todas las subunidades locales relevantes, y por definición abarca dos
variables importantes: (1) la riqueza de especies, y (2) la abundancia
relativa de especies (Carmona-Galindo \& Carmona, 2013). Para calcular
la diversidad alpha se utiliza el indice de Fisher (Fisher, Corbet, \&
Williams, 1943), el índice de Simpson (E. H. Simpson, 1949), y el índice
de Shannon-Wiener (Shannon, n.d.). A fin de determinar la diversidad
alpha se utilizaron métodos como la Entropia de Shannon H1, que calcula
el grado de desorden en la muestra, el índice de concentración de
Simpson, que calcula la probabilidad de que dos individuos seleccionados
aleatoriamente puedan ser de la misma especie. Ademas, se empleó la
serie de números de diversidad de Hill, la fórmula de la entropía de
Rengi y el índice de equidad de Pielou.

Para determinar la rarefaccion se comparó los índices de diversidad
entre hábitats en base a un mismo número de individuos, donde el hábitat
con el mayor número de individuos se submuestreo sin remplazo
aleatoriamente y con múltiples ejecuciones para generar un índice
promedio que se pueda comparar con el índice de otro hábitat en base a
un mismo número de individuos. El resultado de esta técnica resulto con
una curva rarificada de valores del índice de diversidad que disminuye
conforme con el muestreo sin remplazo del número de
individuos(Carmona-Galindo \& Carmona, 2013)(ver figura
\ref{fig:Curva_rarefaccion; acumulacion_especies_individuos}).

\subsection{Analisis de diversidad
beta}\label{analisis-de-diversidad-beta}

La diversidad beta, de acuerdo con (Whittaker, 1960), se define como el
diferencial entre la diversidad de un hábitat y la diversidad total de
un paisaje de hábitats. Teniendo en vuenta lo anterior, se utilizo el
metodo hellinger para determinar cuales son las especies que contribuyen
a la diversidad beta, es decir, las especies que no se encuentran
compartidas entre los cuadrantes.

\subsection{Análisis de ordenación simple (no restringida) y canónica
(restringida)}\label{anuxe1lisis-de-ordenaciuxf3n-simple-no-restringida-y-canuxf3nica-restringida}

Se aplicaron los análisis de componentes principales o PCA a los datos
de las variables ambientale para determianar la ordenacion no
restringida, donde se tomó en cuenta especificamente las variables del
suelo. y para determinar la ordenacion restringida se exploró de manera
explícita las relaciones entre una matriz de respuesta y una matriz
explicativa con los análisis de redundancia o RDA, lo cual combina la
regresión y el análisis de componentes principales (PCA), por ejemplo,
busca tendencias en la matriz de comunidad restringiéndolas a la matriz
ambiental.

\subsection{Ecología espacial}\label{ecologuxeda-espacial}

En ecología espacial se utilizó la matriz transformada de hellinger y la
matriz ambiental para crear un cuadro de vecindad y ver como se
autocorrelacionan los sitios. Se genera un correlograma para la variable
que queremos estudiar mediante la función `sp.correlogram' y para varias
variables como la abundancia de especies y las variables ambientales.
También, se utilizaron otros métodos como la prueba Mantel con matrices
de distancia para autocorrelación espacial con y sin tendencia, y el I
de Moran con una matriz de abundancia de especies transformada sin
tendencia, lo cual se aplica a variables ambientales para obtener los
datos de autocorrelación, distribución de especies y variables en los
sitios de muestreo.

\section{Resultados}\label{resultados}

En toda la parcela, se registró un total de 2029 pertenecientes a 5
especies. La riqueza por cuadro fue de 4 especies y la mediana de la
abundancia por cuadro fue de 39 individuos. La especie más abundante fue
\emph{Pouteria reticulata}, con 1084 individuos, y la menos abundante
fue \emph{Pouteria fossicola} con 3 individuos. La tabla
\ref{tab:abun_sp} y la figura \ref{fig:abun_sp_q} resume estos
resultados.

\begin{longtable}[]{@{}lr@{}}
\caption{\label{tab:abun_sp}Abundancia por especie de la familia
Sapotaceae}\tabularnewline
\toprule
Latin & n\tabularnewline
\midrule
\endfirsthead
\toprule
Latin & n\tabularnewline
\midrule
\endhead
Pouteria reticulata & 1084\tabularnewline
Chrysophyllum argenteum & 711\tabularnewline
Chrysophyllum cainito & 171\tabularnewline
Pouteria stipitata & 60\tabularnewline
Pouteria fossicola & 3\tabularnewline
\bottomrule
\end{longtable}

\begin{figure}
\centering
\includegraphics{manuscrito_files/figure-latex/unnamed-chunk-3-1.pdf}
\caption{\label{fig:abun_sp_q}Abundancia por especie por quadrat}
\end{figure}

\begin{figure}
\centering
\includegraphics{manuscrito_files/figure-latex/unnamed-chunk-4-1.pdf}
\caption{\label{fig:P13}Diagrama de cajas de la abundancia y riqueza
segun habitats}
\end{figure}

la distribucion de la riqueza numerica de especies de la familia
Sapotaceae sigue un patron homogeneo, lo cual los agregados de riqueza
maxima estan distribuidos en casi todo el area. (ver Figura
\ref{fig:mapa_cuadros_riq_mi_familia})

\begin{figure}
\centering
\includegraphics[width=0.50000\textwidth]{mapa_cuadros_riq_mi_familia.png}
\caption{Distribucion de la riqueza de la familia
Sapotaceae\label{fig:mapa_cuadros_riq_mi_familia}}
\end{figure}

Las variables ambientales pH y pendiente media presentaron asociacion
con la familia de plantas\ldots{}, lo cual supone\ldots{} (ver figuras
\ref{fig:mapa_cuadros_pH} y \ref{fig:mapa_cuadros_pendiente}).

\begin{figure}
\centering
\includegraphics[width=0.50000\textwidth]{mapa_cuadros_ph.png}
\caption{Distribucion del pH\label{fig:mapa_cuadros_pH}}
\end{figure}

\begin{figure}
\centering
\includegraphics[width=0.50000\textwidth]{mapa_cuadros_pendiente.png}
\caption{Distribucion de las pendientes(en
grados)\label{fig:mapa_cuadros_pendiente}}
\end{figure}

la abundancia de la familia sapotaceae solo presenta correlacion con la
abundacia global, mientras que la riqueza tiene correlacion con la
presencia de cobre y nitrogeno en el suelo, lo que sugiere\ldots{} (ver
figura \ref{fig:p_cor_suelo_ar}).

\begin{figure}
\centering
\includegraphics{manuscrito_files/figure-latex/unnamed-chunk-5-1.pdf}
\caption{\label{fig:p_cor_suelo_ar}correlacion de las variables del
suelo}
\end{figure}

las variables ambientales numericas y nominales presentan un patron (ver
figuras \ref{fig:mapas_variables_ambientales_numericas} y
\ref{fig:mapas_variables_ambientales_nominales}).

\begin{figure}
\centering
\includegraphics[width=1.00000\textwidth]{mapas_variables_ambientales_numericas.png}
\caption{variables ambientales
numericas\label{fig:mapas_variables_ambientales_numericas}}
\end{figure}

\begin{figure}
\centering
\includegraphics{mapas_variables_ambientales_nominales_tmap.png}
\caption{variables ambientales
nominales\label{fig:mapas_variables_ambientales_nominales}}
\end{figure}

El indice de similaridad de Jaccard muestra que el sitio 1 y 2 comparten
un 100\% de sus especies, por lo que ambos sitios comparten 3 especies y
no tienen especies exclusivas (ver figura
\ref{fig:similaridad_jaccard}).

\begin{figure}
\centering
\includegraphics{medicion_asociacion_jaccard.png}
\caption{Similaridad de Jaccard(color fucsia (magenta, rosa) significa
``corta distancia=muy similares'', y cian (celeste) significa ``gran
distancia=poco similares'')\label{fig:similaridad_jaccard}}
\end{figure}

La correlcion entre las variables geomorfologicas con la abundancia y
riqueza\ldots{}(ver figura
\ref{fig:matriz_correlacion_geomorf_abun_riq_spearman}).

\begin{figure}
\centering
\includegraphics{medicion_asociacion_jaccard.png}
\caption{Panel de correlacion de Spearman entre los datos de la
comunidad y las variables
geomorfologicas\label{fig:matriz_correlacion_geomorf_abun_riq_spearman}}
\end{figure}

Las pruebas de correlación entre los grupos 1 y 2 formulados por
upgma\ldots{}(ver figura \ref{fig:grupos_upgma}).

\begin{figure}
\centering
\includegraphics[width=1.00000\textwidth]{actualizacion2_grupos_upgma.png}
\caption{Diagramas de caja de las variables que tuvieron un efecto,
segun las pruebas de igualdad de medias\label{fig:grupos_upgma}}
\end{figure}

La repartición de sitios en los grupos formulados por enlace
upgm\ldots{}(ver figura \ref{fig:mapa_upgma_k2}).

\begin{figure}
\centering
\includegraphics{mapa_upgma_k2.png}
\caption{Mapa en el que se presenta la distribucion de sitios en los
grupos formulados por enlace upgma\label{fig:mapa_upgma_k2}}
\end{figure}

Las especies indicadoras fueron\ldots{}(verificar analisis de
agrupamiento 4)

Análisis de especies indicadoras mediante IndVal Association function:
IndVal.g Significance level (alpha): 0.05

Total number of species: 5 Selected number of species: 2 Number of
species associated to 1 group: 2

List of species associated to each combination:

Group 1 \#sps. 1 A B stat p.value\\
Chrysophyllum argenteum 0.6565 1.0000 0.81 0.005 **

Group 2 \#sps. 1 A B stat p.value\\
Pouteria reticulata 0.6505 1.0000 0.807 0.001 ***

grado de correlacion que existe en cado uno de los indices\ldots{} (ver
figura \ref{fig:correlacion_indices}).

\begin{figure}
\centering
\includegraphics[width=1.00000\textwidth]{correlacion_pearson.png}
\caption{Grado de correlacion entre cado uno de los indices. N0: riqueza
de especies; H: entropia de Shannon; Hb2: entropia de Shannon con 2 como
base del logaritmo; N1 y N2: Numeros de Hill; N1b2: Numero de Hill 1 en
base log 2; J: equidad de Pielou; E10 y E20: ratios de Hill 1 y 2
\label{fig:correlacion_indices}}
\end{figure}

La riqueza de la familia Sapotaceae aumenta en función del contenido de
hierro, nitrógeno y cobre, tambien aunmenta con la equidad (ver figura
\ref{fig:correlacion_diversidad_equidad}).

\begin{figure}
\centering
\includegraphics[width=1.00000\textwidth]{correlacion_diversidad_equidad_actualizado.png}
\caption{Correlacion entre diversidad/equidad y algunas de las variables
ambientales \label{fig:correlacion_diversidad_equidad}}
\end{figure}

Sitios que tienen mayores valores de equidad (azul y verde)\ldots{} (ver
figura \ref{fig:grafico_niveles_equidad}).

\begin{figure}
\centering
\includegraphics{grafico_niveles_equidad.png}
\caption{Grafico dque presenta los valores de equidad por sitos
\label{fig:grafico_niveles_equidad}}
\end{figure}

Curva de rarefaccion de los sitios, teniendo en cuenta la riqueza y la
abundancia\ldots{}(ver figura \ref{fig:Curva_rarefaccion}). analizar
Analisis de diversidad 1

\begin{figure}
\centering
\includegraphics{Curva_rarefaccion.png}
\caption{Curva de rarefaccion de los sitios
\label{fig:Curva_rarefaccion}}
\end{figure}

asymptotic diversity estimates along with related statistics. Observed
Estimator Est\_s.e. 95\% Lower 95\% Upper Species Richness 5.000 5.000
0.217 5.000 5.481 Shannon diversity 2.786 2.789 0.045 2.786 2.876
Simpson diversity 2.403 2.404 0.038 2.403 2.478

Acumulacion de especies en funcion de numeros de individuos\ldots{}(ver
figura \ref{fig:acumulacion_especies_individuos}).

\begin{figure}
\centering
\includegraphics{acumulacion_especies_individuos.png}
\caption{Grafico de acumulacion de especies en funcion de numeros de
individuos \label{fig:acumulacion_especies_individuos}}
\end{figure}

Valores de diversidad beta por cada una de las especies y cuales son su
contribucion (comparar cueles son las especies que contribuyen mas a la
diveridad beta) especies\_contribuyen\_betadiv Chrysophyllum argenteum
Chrysophyllum cainito Pouteria stipitata 0.2504234 0.3147978 0.2658814

\$sitios\_contribuyen\_betadiv {[}1{]} ``9'' ``40''

Estos sitios presentan contribucion a la diversidad beta por la
incidencia de algunas variables ambientales (habitat, y variables
numericas\ldots{})(ver figura
\ref{fig:mapas_variables_ambientales_numericas} y
\ref{fig:mapas_variables_ambientales_nominales}).

Componentes príncipales de la varianza en las variables de suelo y
geomorfología en BCI.En estos gráficos se incluye el comportamiento de
la varianza explicada, predecido por el modelo de bara quebrada,
representado por la línea roja formando la curva. (La escala denominada
``Inertia'' representa la suma de los cuadrados de toda la varianza)
(ver figura \ref{fig:env_suelo_pca}).

\begin{figure}
\centering
\includegraphics{env_suelo_pca.png}
\caption{grafico de los componentes principales de la varianza en las
variables suelo y geomorfologia en BCI \label{fig:env_suelo_pca}}
\end{figure}

Se observa que las variables nitrógeno, fósforo y pH aportan la mayor
parte de la varianza explicada. La relación entre las variables se
encuentra debidamente representada en el recuadro del escalamiento 2,
por medio delos ángulos que forman sus vectores (ver figura
\ref{fig:Biplot_PCA_escalamiento}).

\begin{figure}
\centering
\includegraphics[width=1.00000\textwidth]{Biplot_PCA_escalamiento_actualizado.png}
\caption{Biplots generados en el PCA de las viariables de suelo
\label{fig:Biplot_PCA_escalamiento}}
\end{figure}

El escalamiento 1, muestra muchos de los cuadrantes dispuestos alrededor
del origen formado por los ejes, lo que indica una contribución a la
varianza relativamente equitativa por parte de las especies. Sin
embargo, aparecen también unos cuantos cuadrantes con valores atípicos y
más alejados. Se nota como las especies (mencionar especies) presentan
una contribución desproporcionada a la varianza total, en comparación
con el resto de las especies (ver figura \ref{fig:PCA_comunidad}).

\begin{figure}
\centering
\includegraphics[width=1.00000\textwidth]{PCA_comunidad_actualizado.png}
\caption{Biplots generados en el PCA de las viariables de suelo
\label{fig:PCA_comunidad}}
\end{figure}

Biplot del análisis de correspondencia de los datos de abundancia de las
especies de Sapotaceae (ver figura
\ref{fig:Analisis_de_correspondencia}).

\begin{figure}
\centering
\includegraphics[width=1.00000\textwidth]{analisis_de_correspondencia_actualizado.png}
\caption{Biplot del analisis de correspondencia de los datos de
abundancia de las especies de Sapotaceae
\label{fig:Analisis_de_correspondencia}}
\end{figure}

(ver figura \ref{fig:PCoA _promedios_especies}).

\begin{figure}
\centering
\includegraphics[width=1.00000\textwidth]{Promedios_ponderados.png}
\caption{PCoA con promedios ponderados de especies
\label{fig:PCoA _promedios_especies}}
\end{figure}

(ver figura \ref{fig:PCoA _promedios_especies}).

\begin{figure}
\centering
\includegraphics[width=1.00000\textwidth]{especies_ecologia_espacial.png}
\caption{Abundancia de especies (matriz de comuniddad transformada)
\label{fig:PCoA _promedios_especies}}
\end{figure}

\section{Discusión}\label{discusiuxf3n}

\section{Agradecimientos}\label{agradecimientos}

\section{Información de soporte}\label{informaciuxf3n-de-soporte}

\ldots

\section{\texorpdfstring{\emph{Script}
reproducible}{Script reproducible}}\label{script-reproducible}

\ldots

\section*{Referencias}\label{referencias}
\addcontentsline{toc}{section}{Referencias}

\hypertarget{refs}{}
\hypertarget{ref-jose_ramon_martinez_batlle_2020_4402362}{}
Batlle, J. R. M. (2020). biogeografia-master/scripts-de-analisis-BCI:
Long coding sessions (Version v0.0.0.9000).
\url{https://doi.org/10.5281/zenodo.4402362}

\hypertarget{ref-brocard2011numerical}{}
Brocard, D., Gillet, F., \& Legendre, P. (2018). Numerical ecology with
r. \emph{Springer Nature}, \emph{Second Edition}, 52--66. Retrieved from
\url{https://doi.org/10.1007/978-3-319-71404-2}

\hypertarget{ref-carmona2013diversidad}{}
Carmona-Galindo, V. D., \& Carmona, T. V. (2013). La diversidad de los
análisis de diversidad la diversidad de los analisis de diversidad
{[}the diversity of diversity analyses{]}. \emph{Bioma}.

\hypertarget{ref-caceres2009associations}{}
Cáceres, M. D., \& Legendre, P. (2009). Associations between species and
groups of sites: Indices and statistical inference. \emph{Ecology},
\emph{90}(12), 3566--3574.

\hypertarget{ref-condit2012thirty}{}
Condit, R. A. y H., Richard y Chisholm. (n.d.). Treinta años de censo
forestal en barro colorado y la importancia de la inmigración para
mantener la diversidad. \emph{PloS One}.

\hypertarget{ref-condit2017demographic}{}
Condit, R. y L., Richard y P ~'e rez. (n.d.). Tendencias demográficas y
clima durante 35 años en la parcela de 50 ha de barro colorado.
\emph{Forest Ecosystems}.

\hypertarget{ref-dufrene1997species}{}
Conjuntos de especies y especies indicadoras: La necesidad de un enfoque
asimétrico flexible. (n.d.). \emph{Monografías Ecológicas}.

\hypertarget{ref-indicspecies}{}
De Caceres, M., \& Legendre, P. (2009). Associations between species and
groups of sites: Indices and statistical inference. In \emph{Ecology}.
Retrieved from \url{http://sites.google.com/site/miqueldecaceres/}

\hypertarget{ref-fisher1943relation}{}
Fisher, R. A., Corbet, A. S., \& Williams, C. B. (1943). The relation
between the number of species and the number of individuals in a random
sample of an animal population. \emph{The Journal of Animal Ecology},
42--58.

\hypertarget{ref-croat1978flora}{}
\emph{Flora de la isla barro colorado}. (n.d.). Stanford University
Press.

\hypertarget{ref-leigh1990barro}{}
Isla barro colorado y biología tropical. (n.d.). \emph{Cuatro Bosques
Neotropicales}.

\hypertarget{ref-diversityanalysis}{}
Kindt, R., \& Coe, R. (2005). \emph{Tree diversity analysis. a manual
and software for common statistical methods for ecological and
biodiversity studies}. Retrieved from
\url{http://www.worldagroforestry.org/output/tree-diversity-analysis}

\hypertarget{ref-legendre2001ecologically}{}
Legendre, P., \& Gallagher, E. D. (2001). Ecologically meaningful
transformations for ordination of species data. \emph{Oecologia},
\emph{129}(2), 271--280.

\hypertarget{ref-martinez2020importancia}{}
Martínez-Sovero, G., Iglesias-Osores, S., \& Villena-Velásquez, J. J.
(2020). Importancia de la familia sapotaceae en madre de dios, perú.
\emph{Manglar}, \emph{17}(4), 287.

\hypertarget{ref-martinez2021diversidad}{}
Martínez-Sovero, G., Iglesias-Osores, S., Muñoz-Chavarry, P.,
Seminario-Cunya, A., Alva-Mendoza, D., \& Villena-Velásquez, J. (2021).
Diversidad y estructura de sapotaceae en bosques amazónicos de madre de
dios, perú. \emph{Ciencia Amazónica (Iquitos)}, \emph{9}(1), 59--72.

\hypertarget{ref-vegan}{}
Oksanen, J., Blanchet, F. G., Friendly, M., Kindt, R., Legendre, P.,
McGlinn, D., \ldots{} Wagner, H. (2019). \emph{Vegan: Community ecology
package}. Retrieved from \url{https://CRAN.R-project.org/package=vegan}

\hypertarget{ref-perez2005metodologia}{}
Pérez, R., Aguilar, S., Condit, R., Foster, R., Hubbell, S., \& Lao, S.
(2005). Metodologia empleada en los censos de la parcela de 50 hectareas
de la isla de barro colorado, panamá. \emph{Centro de Ciencias
Forestales Del Tropico (CTFS) Y Instituto Smithsonian de Investigaciones
Tropicales (STRI)}, 1--24.

\hypertarget{ref-Restudio}{}
R Core Team. (2020). \emph{R: A language and environment for statistical
computing}. Retrieved from \url{https://www.R-project.org/}

\hypertarget{ref-rodriguez2020scolytinae}{}
Rodríguez-Flores, W., \& Barrios, H. (2020). Scolytinae y platypodinae
(coleoptera: Curculionidae) de la isla barro colorado, panamá.
\emph{Scientia}, \emph{30}(1), 15--52.

\hypertarget{ref-shannon1948mathematical}{}
Shannon, C. E. (n.d.). Una teoría matemática de la comunicación.
\emph{El Diario Técnico Del Sistema Bell}.

\hypertarget{ref-simpson1949measurement}{}
Simpson, E. H. (1949). Measurement of diversity. \emph{Nature},
\emph{163}(4148), 688--688.

\hypertarget{ref-smedmark2007boreotropical}{}
Smedmark, A. A., Jenny EE y Anderberg. (n.d.). La migración
boreotropical explica la hibridación entre linajes geográficamente
distantes en el clado pantropical sideroxyleae (sapotaceae).
\emph{American Journal of Botany}.

\hypertarget{ref-whittaker1960vegetation}{}
Whittaker, R. H. (1960). Vegetation of the siskiyou mountains, oregon
and california. \emph{Ecological Monographs}, \emph{30}(3), 279--338.

\hypertarget{ref-tidyverse}{}
Wickham, H. (2017). \emph{Tidyverse: Easily install and load the
'tidyverse'}. Retrieved from
\url{https://CRAN.R-project.org/package=tidyverse}

\hypertarget{ref-windsorestructura}{}
Windsor, D., FOSTER, R., BROKAW, N., Leigh, E., Rand, A., \& others.
(n.d.). \emph{Estructura e historia de la vegetación de la isla barro
coloradoecología de un bosque tropical: Ciclos estacionales y cambios a
largo plazo}.




\newpage
\singlespacing 
\end{document}
